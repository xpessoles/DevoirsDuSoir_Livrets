\subsection{Déterminer les caractéristiques d'un solide ou d'un ensemble de solides indéformables}

\renewcommand{\repExo}{../../ExercicesCompetences/B2_ProposerModele/B2_10_CartacteristiquesSolides}

\renewcommand{\td}{40_Parallelepipede}
\graphicspath{{\repStyle/png/}{\repExo/\td/images/}}
\input{\repExo/\td/\td.tex}

\renewcommand{\td}{41_Parallelepipede}
\graphicspath{{\repStyle/png/}{\repExo/\td/images/}}
\input{\repExo/\td/\td.tex}

\renewcommand{\td}{42_Cylindre}
\graphicspath{{\repStyle/png/}{\repExo/\td/images/}}
\input{\repExo/\td/\td.tex}

\renewcommand{\td}{43_Cylindre}
\graphicspath{{\repStyle/png/}{\repExo/\td/images/}}
\input{\repExo/\td/\td.tex}

\renewcommand{\td}{44_Disque}
\graphicspath{{\repStyle/png/}{\repExo/\td/images/}}
\input{\repExo/\td/\td.tex}

\renewcommand{\td}{45_Disque}
\graphicspath{{\repStyle/png/}{\repExo/\td/images/}}
\input{\repExo/\td/\td.tex}

\renewcommand{\td}{50_BancBalafre}
\graphicspath{{\repStyle/png/}{\repExo/\td/images/}}
\input{\repExo/\td/\td.tex}