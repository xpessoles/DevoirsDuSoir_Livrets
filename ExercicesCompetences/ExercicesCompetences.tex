\documentclass[10pt,fleqn]{article} % Default font size and left-justified equations
\usepackage[%
    pdftitle={Exercices de SII},
    pdfauthor={Xavier Pessoles}]{hyperref}

\input{../../Style/packages}
\input{../../Style/new_style}
\input{../../Style/macros_SII}
\input{../../Style/environment}
\usepackage{../../Style/UPSTI_pedagogique}

\newcommand{\macrocomp}{macro_competences}
\newcommand{\comp}{competences}
\newcommand{\td}{fichier_td}
\newcommand{\repExo}{dossier}
\newcommand{\repStyle}{../../Style}




\def\xxYCartouche{-2.25cm}
\def\xxYongletGarde{.5cm}
\def\xxYOnget{.9cm}


\begin{document}

\def\xxcompetences{}
\def\xxfigures{}

\graphicspath{{\repStyle/png/}}



\setlength{\columnseprule}{.1pt}

%% MODELISER
%% Page de garde
\input{../../Style/Entete_Modeliser}
%%\input{../../Style/pagegarde_livret_exos}
%%\vspace{2cm}
%\pagestyle{fancy}
%\thispagestyle{plain}
%
%%% Sujets
%\proffalse
\section{Mettre en œuvre une démarche de résolution analytique}
\begin{multicols}{2}
\subsection{C2-06 -- Déterminer les relations entre les grandeurs géométriques ou cinématiques}

% Lois ES Géométriques
\renewcommand{\repExo}{../../ExercicesCompetences/C2_MettreEnOeuvreDemarche/C2_06_DeterminerLoisES}

\renewcommand{\td}{10_PompePalette}
\graphicspath{{\repStyle/png/}{\repExo/\td/images/}}
\input{\repExo/\td/\td.tex}

\renewcommand{\td}{11_PompePistonsRadiaux}
\graphicspath{{\repStyle/png/}{\repExo/\td/images/}}
\input{\repExo/\td/\td.tex}

\renewcommand{\td}{12_BielleManivelle}
\graphicspath{{\repStyle/png/}{\repExo/\td/images/}}
\input{\repExo/\td/\td.tex}

\renewcommand{\td}{13_TransfoMouvement}
\graphicspath{{\repStyle/png/}{\repExo/\td/images/}}
\input{\repExo/\td/\td.tex}

\renewcommand{\td}{14_Sympact}
\graphicspath{{\repStyle/png/}{\repExo/\td/images/}}
\input{\repExo/\td/\td.tex}

\renewcommand{\td}{15_SympactGalet}
\graphicspath{{\repStyle/png/}{\repExo/\td/images/}}
\input{\repExo/\td/\td.tex}

\renewcommand{\td}{16_Poussoir}
\graphicspath{{\repStyle/png/}{\repExo/\td/images/}}
\input{\repExo/\td/\td.tex}

\renewcommand{\td}{17_4Barres}
\graphicspath{{\repStyle/png/}{\repExo/\td/images/}}
\input{\repExo/\td/\td.tex}

\renewcommand{\td}{18_Maxpid}
\graphicspath{{\repStyle/png/}{\repExo/\td/images/}}
\input{\repExo/\td/\td.tex}

\renewcommand{\td}{19_Graham}
\graphicspath{{\repStyle/png/}{\repExo/\td/images/}}
\input{\repExo/\td/\td.tex}

\renewcommand{\td}{20_VariateurBilles}
\graphicspath{{\repStyle/png/}{\repExo/\td/images/}}
\input{\repExo/\td/\td.tex}

\renewcommand{\td}{54_FauteuilRoulant}
\graphicspath{{\repStyle/png/}{\repExo/\td/images/}}
\input{\repExo/\td/\td.tex}
%\subsection{Déterminer les relations entre les grandeurs géométriques ou cinématiques}

% Transmetteurs de puissance
\renewcommand{\repExo}{../../ExercicesCompetences/C2_MettreEnOeuvreDemarche/C2_06_Transmetteurs}

\renewcommand{\td}{21_TrainSimple}
\graphicspath{{\repStyle/png/}{\repExo/\td/images/}}
\input{\repExo/\td/\td.tex}

\renewcommand{\td}{22_TrainSimple}
\graphicspath{{\repStyle/png/}{\repExo/\td/images/}}
\input{\repExo/\td/\td.tex}

\renewcommand{\td}{23_TrainSimple}
\graphicspath{{\repStyle/png/}{\repExo/\td/images/}}
\input{\repExo/\td/\td.tex}

\renewcommand{\td}{24_TrainSimple}
\graphicspath{{\repStyle/png/}{\repExo/\td/images/}}
\input{\repExo/\td/\td.tex}

\renewcommand{\td}{25_Cheville}
\graphicspath{{\repStyle/png/}{\repExo/\td/images/}}
\input{\repExo/\td/\td.tex}

\renewcommand{\td}{26_RoueMotrice}
\graphicspath{{\repStyle/png/}{\repExo/\td/images/}}
\input{\repExo/\td/\td.tex}

%\subsection{Déterminer les actions mécaniques en dynamique dans le cas où le mouvement est imposé.}
\renewcommand{\repExo}{../../ExercicesCompetences/C2_MettreEnOeuvreDemarche/C2_08_TorseurDynamique}

\renewcommand{\td}{01_T}
\graphicspath{{\repStyle/png/}{\repExo/\td/images/}}
\input{\repExo/\td/\td.tex}

\renewcommand{\td}{02_R}
\graphicspath{{\repStyle/png/}{\repExo/\td/images/}}
\input{\repExo/\td/\td.tex}

\renewcommand{\td}{03_TT}
\graphicspath{{\repStyle/png/}{\repExo/\td/images/}}
\input{\repExo/\td/\td.tex}

\renewcommand{\td}{04_RR}
\graphicspath{{\repStyle/png/}{\repExo/\td/images/}}
\input{\repExo/\td/\td.tex}

\renewcommand{\td}{05_RT}
\graphicspath{{\repStyle/png/}{\repExo/\td/images/}}
\input{\repExo/\td/\td.tex}

\renewcommand{\td}{06_TR}
\graphicspath{{\repStyle/png/}{\repExo/\td/images/}}
\input{\repExo/\td/\td.tex}

\renewcommand{\td}{07_RR3D}
\graphicspath{{\repStyle/png/}{\repExo/\td/images/}}
\input{\repExo/\td/\td.tex}

\renewcommand{\td}{08_RR3D}
\graphicspath{{\repStyle/png/}{\repExo/\td/images/}}
\input{\repExo/\td/\td.tex}

\renewcommand{\td}{09_RT_RSG}
\graphicspath{{\repStyle/png/}{\repExo/\td/images/}}
\input{\repExo/\td/\td.tex}

\renewcommand{\td}{50_BancBalafre}
\graphicspath{{\repStyle/png/}{\repExo/\td/images/}}
\input{\repExo/\td/\td.tex}
%
%\renewcommand{\td}{11_PompePistonsRadiaux}
%\graphicspath{{\repStyle/png/}{\repExo/\td/images/}}
%\input{\repExo/\td/\td.tex}
%
%\renewcommand{\td}{12_BielleManivelle}
%\graphicspath{{\repStyle/png/}{\repExo/\td/images/}}
%\input{\repExo/\td/\td.tex}
%
%\renewcommand{\td}{13_TransfoMouvement}
%\graphicspath{{\repStyle/png/}{\repExo/\td/images/}}
%\input{\repExo/\td/\td.tex}
%
%\renewcommand{\td}{14_Sympact}
%\graphicspath{{\repStyle/png/}{\repExo/\td/images/}}
%\input{\repExo/\td/\td.tex}
%
%\renewcommand{\td}{15_SympactGalet}
%\graphicspath{{\repStyle/png/}{\repExo/\td/images/}}
%\input{\repExo/\td/\td.tex}
%
%\renewcommand{\td}{16_Poussoir}
%\graphicspath{{\repStyle/png/}{\repExo/\td/images/}}
%\input{\repExo/\td/\td.tex}
%
%\renewcommand{\td}{17_4Barres}
%\graphicspath{{\repStyle/png/}{\repExo/\td/images/}}
%\input{\repExo/\td/\td.tex}
%
%\renewcommand{\td}{18_Maxpid}
%\graphicspath{{\repStyle/png/}{\repExo/\td/images/}}
%\input{\repExo/\td/\td.tex}

%\subsection{Déterminer la loi de mouvement dans le cas où les efforts extérieurs sont connus}
\renewcommand{\repExo}{../../ExercicesCompetences/C2_MettreEnOeuvreDemarche/C2_09_DeterminerLoiMouvement}

\renewcommand{\td}{01_T}
\graphicspath{{\repStyle/png/}{\repExo/\td/images/}}
\input{\repExo/\td/\td.tex}

\renewcommand{\td}{02_R}
\graphicspath{{\repStyle/png/}{\repExo/\td/images/}}
\input{\repExo/\td/\td.tex}

\renewcommand{\td}{03_TT}
\graphicspath{{\repStyle/png/}{\repExo/\td/images/}}
\input{\repExo/\td/\td.tex}

\renewcommand{\td}{04_RR}
\graphicspath{{\repStyle/png/}{\repExo/\td/images/}}
\input{\repExo/\td/\td.tex}

\renewcommand{\td}{05_RT}
\graphicspath{{\repStyle/png/}{\repExo/\td/images/}}
\input{\repExo/\td/\td.tex}

\renewcommand{\td}{06_TR}
\graphicspath{{\repStyle/png/}{\repExo/\td/images/}}
\input{\repExo/\td/\td.tex}

\renewcommand{\td}{07_RR3D}
\graphicspath{{\repStyle/png/}{\repExo/\td/images/}}
\input{\repExo/\td/\td.tex}

\renewcommand{\td}{08_RR3D}
\graphicspath{{\repStyle/png/}{\repExo/\td/images/}}
\input{\repExo/\td/\td.tex}

\renewcommand{\td}{09_RT_RSG}
\graphicspath{{\repStyle/png/}{\repExo/\td/images/}}
\input{\repExo/\td/\td.tex}

\renewcommand{\td}{46_RR_RSG}
\graphicspath{{\repStyle/png/}{\repExo/\td/images/}}
\input{\repExo/\td/\td.tex}

%\subsection{Déterminer la loi de mouvement dans le cas où les efforts extérieurs sont connus -- TEC}

% Lois ES Géométriques
\renewcommand{\repExo}{../../ExercicesCompetences/C2_MettreEnOeuvreDemarche/C2_09_DeterminerLoiMouvement_TEC}

\renewcommand{\td}{10_PompePalette}
\graphicspath{{\repStyle/png/}{\repExo/\td/images/}}
\input{\repExo/\td/\td.tex}

\renewcommand{\td}{11_PompePistonsRadiaux}
\graphicspath{{\repStyle/png/}{\repExo/\td/images/}}
\input{\repExo/\td/\td.tex}

\renewcommand{\td}{12_BielleManivelle}
\graphicspath{{\repStyle/png/}{\repExo/\td/images/}}
\input{\repExo/\td/\td.tex}

\renewcommand{\td}{13_TransfoMouvement}
\graphicspath{{\repStyle/png/}{\repExo/\td/images/}}
\input{\repExo/\td/\td.tex}

\renewcommand{\td}{14_Sympact}
\graphicspath{{\repStyle/png/}{\repExo/\td/images/}}
\input{\repExo/\td/\td.tex}

\renewcommand{\td}{15_SympactGalet}
\graphicspath{{\repStyle/png/}{\repExo/\td/images/}}
\input{\repExo/\td/\td.tex}

\renewcommand{\td}{16_Poussoir}
\graphicspath{{\repStyle/png/}{\repExo/\td/images/}}
\input{\repExo/\td/\td.tex}

\renewcommand{\td}{17_4Barres}
\graphicspath{{\repStyle/png/}{\repExo/\td/images/}}
\input{\repExo/\td/\td.tex}

\renewcommand{\td}{18_Maxpid}
\graphicspath{{\repStyle/png/}{\repExo/\td/images/}}
\input{\repExo/\td/\td.tex}

%\renewcommand{\td}{19_Graham}
%\graphicspath{{\repStyle/png/}{\repExo/\td/images/}}
%\input{\repExo/\td/\td.tex}
%
%\renewcommand{\td}{20_VariateurBilles}
%\graphicspath{{\repStyle/png/}{\repExo/\td/images/}}
%\input{\repExo/\td/\td.tex}


\end{multicols}


%% Correction %\setcounter{numexo}{0}
\newpage
\proftrue


\section{Mettre en œuvre une démarche de résolution analytique}
\subsection{C2-06 -- Déterminer les relations entre les grandeurs géométriques ou cinématiques}

% Lois ES Géométriques
\renewcommand{\repExo}{../../ExercicesCompetences/C2_MettreEnOeuvreDemarche/C2_06_DeterminerLoisES}

\renewcommand{\td}{10_PompePalette}
\graphicspath{{\repStyle/png/}{\repExo/\td/images/}}
\input{\repExo/\td/\td.tex}

\renewcommand{\td}{11_PompePistonsRadiaux}
\graphicspath{{\repStyle/png/}{\repExo/\td/images/}}
\input{\repExo/\td/\td.tex}

\renewcommand{\td}{12_BielleManivelle}
\graphicspath{{\repStyle/png/}{\repExo/\td/images/}}
\input{\repExo/\td/\td.tex}

\renewcommand{\td}{13_TransfoMouvement}
\graphicspath{{\repStyle/png/}{\repExo/\td/images/}}
\input{\repExo/\td/\td.tex}

\renewcommand{\td}{14_Sympact}
\graphicspath{{\repStyle/png/}{\repExo/\td/images/}}
\input{\repExo/\td/\td.tex}

\renewcommand{\td}{15_SympactGalet}
\graphicspath{{\repStyle/png/}{\repExo/\td/images/}}
\input{\repExo/\td/\td.tex}

\renewcommand{\td}{16_Poussoir}
\graphicspath{{\repStyle/png/}{\repExo/\td/images/}}
\input{\repExo/\td/\td.tex}

\renewcommand{\td}{17_4Barres}
\graphicspath{{\repStyle/png/}{\repExo/\td/images/}}
\input{\repExo/\td/\td.tex}

\renewcommand{\td}{18_Maxpid}
\graphicspath{{\repStyle/png/}{\repExo/\td/images/}}
\input{\repExo/\td/\td.tex}

\renewcommand{\td}{19_Graham}
\graphicspath{{\repStyle/png/}{\repExo/\td/images/}}
\input{\repExo/\td/\td.tex}

\renewcommand{\td}{20_VariateurBilles}
\graphicspath{{\repStyle/png/}{\repExo/\td/images/}}
\input{\repExo/\td/\td.tex}

\renewcommand{\td}{54_FauteuilRoulant}
\graphicspath{{\repStyle/png/}{\repExo/\td/images/}}
\input{\repExo/\td/\td.tex}
%\subsection{Déterminer les relations entre les grandeurs géométriques ou cinématiques}

% Transmetteurs de puissance
\renewcommand{\repExo}{../../ExercicesCompetences/C2_MettreEnOeuvreDemarche/C2_06_Transmetteurs}

\renewcommand{\td}{21_TrainSimple}
\graphicspath{{\repStyle/png/}{\repExo/\td/images/}}
\input{\repExo/\td/\td.tex}

\renewcommand{\td}{22_TrainSimple}
\graphicspath{{\repStyle/png/}{\repExo/\td/images/}}
\input{\repExo/\td/\td.tex}

\renewcommand{\td}{23_TrainSimple}
\graphicspath{{\repStyle/png/}{\repExo/\td/images/}}
\input{\repExo/\td/\td.tex}

\renewcommand{\td}{24_TrainSimple}
\graphicspath{{\repStyle/png/}{\repExo/\td/images/}}
\input{\repExo/\td/\td.tex}

\renewcommand{\td}{25_Cheville}
\graphicspath{{\repStyle/png/}{\repExo/\td/images/}}
\input{\repExo/\td/\td.tex}

\renewcommand{\td}{26_RoueMotrice}
\graphicspath{{\repStyle/png/}{\repExo/\td/images/}}
\input{\repExo/\td/\td.tex}

%\subsection{Déterminer les actions mécaniques en dynamique dans le cas où le mouvement est imposé.}
\renewcommand{\repExo}{../../ExercicesCompetences/C2_MettreEnOeuvreDemarche/C2_08_TorseurDynamique}

\renewcommand{\td}{01_T}
\graphicspath{{\repStyle/png/}{\repExo/\td/images/}}
\input{\repExo/\td/\td.tex}

\renewcommand{\td}{02_R}
\graphicspath{{\repStyle/png/}{\repExo/\td/images/}}
\input{\repExo/\td/\td.tex}

\renewcommand{\td}{03_TT}
\graphicspath{{\repStyle/png/}{\repExo/\td/images/}}
\input{\repExo/\td/\td.tex}

\renewcommand{\td}{04_RR}
\graphicspath{{\repStyle/png/}{\repExo/\td/images/}}
\input{\repExo/\td/\td.tex}

\renewcommand{\td}{05_RT}
\graphicspath{{\repStyle/png/}{\repExo/\td/images/}}
\input{\repExo/\td/\td.tex}

\renewcommand{\td}{06_TR}
\graphicspath{{\repStyle/png/}{\repExo/\td/images/}}
\input{\repExo/\td/\td.tex}

\renewcommand{\td}{07_RR3D}
\graphicspath{{\repStyle/png/}{\repExo/\td/images/}}
\input{\repExo/\td/\td.tex}

\renewcommand{\td}{08_RR3D}
\graphicspath{{\repStyle/png/}{\repExo/\td/images/}}
\input{\repExo/\td/\td.tex}

\renewcommand{\td}{09_RT_RSG}
\graphicspath{{\repStyle/png/}{\repExo/\td/images/}}
\input{\repExo/\td/\td.tex}

\renewcommand{\td}{50_BancBalafre}
\graphicspath{{\repStyle/png/}{\repExo/\td/images/}}
\input{\repExo/\td/\td.tex}
%
%\renewcommand{\td}{11_PompePistonsRadiaux}
%\graphicspath{{\repStyle/png/}{\repExo/\td/images/}}
%\input{\repExo/\td/\td.tex}
%
%\renewcommand{\td}{12_BielleManivelle}
%\graphicspath{{\repStyle/png/}{\repExo/\td/images/}}
%\input{\repExo/\td/\td.tex}
%
%\renewcommand{\td}{13_TransfoMouvement}
%\graphicspath{{\repStyle/png/}{\repExo/\td/images/}}
%\input{\repExo/\td/\td.tex}
%
%\renewcommand{\td}{14_Sympact}
%\graphicspath{{\repStyle/png/}{\repExo/\td/images/}}
%\input{\repExo/\td/\td.tex}
%
%\renewcommand{\td}{15_SympactGalet}
%\graphicspath{{\repStyle/png/}{\repExo/\td/images/}}
%\input{\repExo/\td/\td.tex}
%
%\renewcommand{\td}{16_Poussoir}
%\graphicspath{{\repStyle/png/}{\repExo/\td/images/}}
%\input{\repExo/\td/\td.tex}
%
%\renewcommand{\td}{17_4Barres}
%\graphicspath{{\repStyle/png/}{\repExo/\td/images/}}
%\input{\repExo/\td/\td.tex}
%
%\renewcommand{\td}{18_Maxpid}
%\graphicspath{{\repStyle/png/}{\repExo/\td/images/}}
%\input{\repExo/\td/\td.tex}

%\subsection{Déterminer la loi de mouvement dans le cas où les efforts extérieurs sont connus}
\renewcommand{\repExo}{../../ExercicesCompetences/C2_MettreEnOeuvreDemarche/C2_09_DeterminerLoiMouvement}

\renewcommand{\td}{01_T}
\graphicspath{{\repStyle/png/}{\repExo/\td/images/}}
\input{\repExo/\td/\td.tex}

\renewcommand{\td}{02_R}
\graphicspath{{\repStyle/png/}{\repExo/\td/images/}}
\input{\repExo/\td/\td.tex}

\renewcommand{\td}{03_TT}
\graphicspath{{\repStyle/png/}{\repExo/\td/images/}}
\input{\repExo/\td/\td.tex}

\renewcommand{\td}{04_RR}
\graphicspath{{\repStyle/png/}{\repExo/\td/images/}}
\input{\repExo/\td/\td.tex}

\renewcommand{\td}{05_RT}
\graphicspath{{\repStyle/png/}{\repExo/\td/images/}}
\input{\repExo/\td/\td.tex}

\renewcommand{\td}{06_TR}
\graphicspath{{\repStyle/png/}{\repExo/\td/images/}}
\input{\repExo/\td/\td.tex}

\renewcommand{\td}{07_RR3D}
\graphicspath{{\repStyle/png/}{\repExo/\td/images/}}
\input{\repExo/\td/\td.tex}

\renewcommand{\td}{08_RR3D}
\graphicspath{{\repStyle/png/}{\repExo/\td/images/}}
\input{\repExo/\td/\td.tex}

\renewcommand{\td}{09_RT_RSG}
\graphicspath{{\repStyle/png/}{\repExo/\td/images/}}
\input{\repExo/\td/\td.tex}

\renewcommand{\td}{46_RR_RSG}
\graphicspath{{\repStyle/png/}{\repExo/\td/images/}}
\input{\repExo/\td/\td.tex}

%\subsection{Déterminer la loi de mouvement dans le cas où les efforts extérieurs sont connus -- TEC}

% Lois ES Géométriques
\renewcommand{\repExo}{../../ExercicesCompetences/C2_MettreEnOeuvreDemarche/C2_09_DeterminerLoiMouvement_TEC}

\renewcommand{\td}{10_PompePalette}
\graphicspath{{\repStyle/png/}{\repExo/\td/images/}}
\input{\repExo/\td/\td.tex}

\renewcommand{\td}{11_PompePistonsRadiaux}
\graphicspath{{\repStyle/png/}{\repExo/\td/images/}}
\input{\repExo/\td/\td.tex}

\renewcommand{\td}{12_BielleManivelle}
\graphicspath{{\repStyle/png/}{\repExo/\td/images/}}
\input{\repExo/\td/\td.tex}

\renewcommand{\td}{13_TransfoMouvement}
\graphicspath{{\repStyle/png/}{\repExo/\td/images/}}
\input{\repExo/\td/\td.tex}

\renewcommand{\td}{14_Sympact}
\graphicspath{{\repStyle/png/}{\repExo/\td/images/}}
\input{\repExo/\td/\td.tex}

\renewcommand{\td}{15_SympactGalet}
\graphicspath{{\repStyle/png/}{\repExo/\td/images/}}
\input{\repExo/\td/\td.tex}

\renewcommand{\td}{16_Poussoir}
\graphicspath{{\repStyle/png/}{\repExo/\td/images/}}
\input{\repExo/\td/\td.tex}

\renewcommand{\td}{17_4Barres}
\graphicspath{{\repStyle/png/}{\repExo/\td/images/}}
\input{\repExo/\td/\td.tex}

\renewcommand{\td}{18_Maxpid}
\graphicspath{{\repStyle/png/}{\repExo/\td/images/}}
\input{\repExo/\td/\td.tex}

%\renewcommand{\td}{19_Graham}
%\graphicspath{{\repStyle/png/}{\repExo/\td/images/}}
%\input{\repExo/\td/\td.tex}
%
%\renewcommand{\td}{20_VariateurBilles}
%\graphicspath{{\repStyle/png/}{\repExo/\td/images/}}
%\input{\repExo/\td/\td.tex}



\end{document}



